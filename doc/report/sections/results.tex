\section*{Results}
The mostinteresting finding is the strong correlation between the students per teacher ratio and the Abitur grades. Intuitevly this makes sense, since more teacher should result in smaller class sizes and less stress for the teachers, which makes for a better learning experience. The correlation between education performance and availabilty of teaching personnel is not new to research, but it was usually discussed in the context of university performance (Quelle). It is especially important, since the student teacher ratio got smaller over the current years and the grades went on a steep increase. This marks the importance of having enough teaching personnel for every school.\\\\
It would be easy to say that the schools just need more money, so they can employ more teachers, but that's not the whole story. In this case, thüringen, Sachsen-Anhalt and Brandenburg stand out. For them the correlation between budget and students per teacher is negative. This could be a classic case of the East-West-Gap in Germany. The exact reasons are up to speculation, but a plausible explanation might go like this: Schools get money mainly based on how many children they have -> If certain schools get more children, they might wanna employ more teachers -> Because they don't have enough teachers in these states, they can't -> Schools get more money, but the number of teachers stay the same: Negative correlation.\\\\
The same anomaly can be observed with the repeaters. Here, we think a different phenomenon is accountable for this. Schools in Thüringen and Sachsen-Anhalt rely more and more on Teilzeitkräfte. This means that the overall proportion of teachers increases, while the grades stay the same, or even worsen, becasue there is more flucuation in teaching presonell for a given class. Thus we get a negative correlation.\\\\
That said, we can observe that for every other of the 16 federal states there is a very strong positive correlation, not only between Abitur grades but also the number of repeaters decreases. This means that it not only leads to better grades, but the weak ones won't left behind if enough teachers are available. But money doesn't necessaraly help here. There need to be enough teachers available to employ. From this analysis we conclude that making sure that many teachers are avilable is one of the most important challenges for the education system. Especially because right now there is still more open positions, than teachers that can fill them. The prognosis of the german education minister conference (Kultusministerkonferenz) gives a positive outlook. Their forecast is, that this gap will close in the coming ten years. This means that we can expect a further increase in grades in the future\\\\
It is important to note, having enough teachers is not the only factor at play here. We have shown, that it is one of the most important ones. The german education system has a lot of problems, but as we have shown, it has been solving some of the in the last decade. To say it is just because a grade inflation is happening, as it is a very common argument made, is a simplification and might be entirely wrong. The increasing grades are a result of an increase in competence of the students, facilitated by an improve in the education system, especially a decrease in the student teacher ratio.\\\\
