\section*{Results}
The mostinteresting finding is the strong coreelation between the students per teacher ratio and the Abitur grades. Intuitevly this makes sense, since more teacher should result in smaller class sizes and less stress for the teachers, which makes for a better learning experience. The correlation between education performance and availabilty of teaching personnel is not new to research, but it was usually discussed in the context of university performance (Quelle). It is especially important, since the student teacher ratio got smaller over the current years and the grades went on a steep increase. This marks the importance of having enough teaching personnel for every school.\\\\
It would be easy to say that the schools just need more money, so they can employ more teachers, but that's not the whole story. In this case, thüringen, Sachsen-Anhalt and Brandenburg stand out. For them the correlation between budget and students per teacher is negative. This could be a classic case of the East-West-Gap in Germany. The exact reasons are up to speculation, but a plausible explanation might go like this: Schools get money mainly based on how many children they have -> If certain schools get more children, they might wanna employ more teachers -> Because they don't have enough teachers in these states, they can't -> Schools get more money, but the number of teachers stay the same: Negative correlation.\\\\
The same anomaly can be observed with the repeaters. Here, we think a different phenomenon is accountable for this. Schools in Thüringen and Sachsen-Anhalt rely more and more on Teilzeitkräfte. This means that the overall proportion of teachers increases, while the grades stay the same, or even worsen, becasue there is more flucuation in teaching presonell for a given class. Thus we get a negative correlation.\\\\
That said, we can observe that for every other of the 16 federal states there is a very strong positive correlation, not only between Abitur grades but also the number of repeaters decreases. This means that it not only leads to better grades, but the weak ones won't left behind if enough teachers are available. But money doesn't necessaraly help here. There need to be enough teachers available to employ. From this analysis we conclude that making sure that many teachers are avilable is one of the most important challenges for the education system. As seen by the Studie Lehrerbedarfsdings bums by the german Kultusministerium... Was haben die eigentlich prognostiziert?
