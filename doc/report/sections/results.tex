%!TEX root = ../report_template.tex
\section{Results and Discussion}
The introduced student-to-teacher ratio combines the number of children with the number of teachers, either for each federal state or school type. Thus, this section compares the student-to-teacher ratio with the Abitur grades, repeaters, and budgets. Importantly, the Abitur grades can only be analyzed for the German average, due to the missing representation. Nevertheless, the repeaters and budgets are analyzed for each federal state.

One of the key findings of this analysis is the strong correlation between the average grades across all federal states and the student-to-teacher ratio in German grammar schools. As shown in \autoref{fig:regression-stt-grade}, the relationship between both is nearly linear and contains neither clusters nor outliers. Hence, a smaller student-to-teacher ratio strongly correlates to better Abitur grades, with a Pearson correlation coefficient of $0.98$.

\begin{figure}[ht]
    \centering
    \includegraphics{fig/fig_correlation_grades_students_per_teacher.pdf}
    \caption{Linear regression on the student-to-teacher ratio by average Abitur grade. The resulting regression line (\textcolor{TUred}{\rule[-0.2ex]{0.5em}{2pt} \rule[-0.2ex]{0.5em}{2pt}}) is calculated over the aggregated average overall grammar schools (\textcolor{TUlightblue}{\tikz\draw[fill={TUlightblue}] (0,0) circle (0.25em);}) in Germany.}
    \label{fig:regression-stt-grade}
\end{figure}

This result confirms the initial hypothesis of the big influence of the student-to-teacher ratio and is consistent with studies in other countries \cite{kasau_onesmus_mulei_pupil-teacher_2016,koc_impact_2015,dickson_economic_1984}.

However, the Abitur grades give only an insight into grammar schools. Thus, the correlations with repeaters and budgets are shown in \autoref{fig:heatmap_correlation_students_per_teacher_repeaters_budget}. It presents a visualization of the Pearson correlation coefficients, analyzing the relationship between the number of children per teacher and the average number of repeaters, as well as the educational budget per child. Therefore, the Pearson correlation coefficients for each state are normalized to the used color map scale.

\begin{figure}[ht]
    \centering
    \includegraphics{fig/fig_heatmap_correlation_students_per_teacher_repeaters_budget.pdf}
    \caption{Pearson correlation coefficients between the student-to-teacher ratio and the relative repeater count (left) and the inflation-adjusted average budget per child (right). \textcolor{red}{Red} indicates positive, \textcolor{gray}{light gray} neutral, \textcolor{TUdark}{gray} missing, and \textcolor{blue}{blue} negative correlations between the variables.}
    \label{fig:heatmap_correlation_students_per_teacher_repeaters_budget}
\end{figure}

The findings represented in \autoref{fig:heatmap_correlation_students_per_teacher_repeaters_budget} support a  positive correlation between the student-to-teacher ratio and the number of repeaters in most federal states. In contrast, they indicate a negative correlation between the student-to-teacher ratio  and budget per child for the same states. Furthermore, there is a big difference between \emph{old} and \emph{new} federal states for both correlations. Especially, the results for Thüringen and Brandenburg diverge from the average in both maps. Rather, Mecklenburg-Vorpommern differs most in the budget per child from the other states.

However,  an increase in the number of students in the new federal states can explain the different correlations in the budget per child\footref{footnote:teachers-children}. Since the per-child budget has increased in all states\footref{footnote:budget}, the schools got more cash in total. This results in more vacancies at schools \cite{kultusminister_konferenz_lehrkrafteeinstellungsbedarf_2023} and other investments, like digitalization and maintenance of schools \cite{bundesministerium_fur_bildung_und_forschung_fortschrittsbericht_2022}. But in the short term, there will be more teachers needed as available \cite{kultusminister_konferenz_lehrkrafteeinstellungsbedarf_2023}. As \autoref{fig:teacher_contracts} indicates, this leads to a high variance in the distribution of teacher contract types, especially for the new federal states. This results in a higher student-to-teacher ratio with the same budget.

In contrast, the anomaly for the repeaters involves more aspects of the educational system, like the different curricula or conditions for repeating a class. The importance attached to repetition varies from federal state to federal state and cannot necessarily be used as an measure of educational quality \cite{klemm_klassenwiederholungen_2009}. Hence, a complete explanation for these results involves more datasets and effects, which are not analyzed in this paper.

Moreover, all other states indicate that a higher budget per child results in a smaller student-to-teacher ratio. As a result, the number of repeaters decreases with the student-to-teacher ratio.