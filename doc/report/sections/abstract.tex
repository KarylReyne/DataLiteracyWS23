% The reasons for the increase in Abitur grades have been a highly discussed topic in German society. Often, the reason for this is attributed to grade inflation. In this paper, we will show why this is not the case. Student competence has increased. One of the most important factors is the student-teacher ratio. We will show that this metric has a high correlation with the Abitur grades.

% Many mathematicians claiming the existence of grade inflation, due to easier exams.

The steady increase in Abitur grades indicates grade inflation in German schools. This study analyzes the factors influencing the children's performance in the German school system through an exploratory data analysis. Examining variables such as student numbers, teacher ratios, repeaters, and budget per child, the research aims to both describe and correlate these factors. The findings challenge the idea that grade inflation is the only cause of rising grades and emphasize the importance of having a small students-per-teacher ratio.