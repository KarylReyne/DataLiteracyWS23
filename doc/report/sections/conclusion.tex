%!TEX root = ../report_template.tex
\section{Conclusion}
In summary, this exploratory data analysis found correlations between the ratio of children and teachers, Abitur grades, repeaters, and budget per child. These indicate the importance of employing enough teachers, to increase children's performance. Prognoses are  showing that there are still more open positions than teachers that can fill them \cite{kultusminister_konferenz_lehrkrafteeinstellungsbedarf_2023}. Fortunately, they predict that this gap will eventually close in the coming decade. 

Furthermore, this gap of missing teachers has an influence on the Abitur grades. It is evident from the observed correlations that the grades should get worse in the short term and increase in the long term, if there are no other factors influencing the Abitur grades. All in all, this analysis shows that grades at schools do not only rise due to grade inflation, but are also influenced by other positive factors.

% It is important to note that having enough teachers is not the only factor responsible for the rising Abitur grades. Nonetheless, it is one of the most important ones. While the German education system faces several challenges, this analysis illustrates that it has effectively addressed certain issues over the past decade and is poised to continue resolving them in the future. The increasing grades are a result of an increase in the competence of the students, facilitated by an improvement in the education system, especially a decrease in the student-teacher ratio.

% We have introduced a new approach to explaining the increasing Abitur grades. There is a very strong correlation between the student-teacher ratio and the Abitur grades. Additionally, we also found a negative correlation between this ratio and the repeater number. This means that the number of teachers not only has a positive impact on the grades but also on the more challenged students. Improving the budget does not necessarily help. It is essential for the German education system to have enough teachers available.

% Todo Reuse 
% Grade inflation in the Abitur grades has not been scientifically proven so far. What has been proven is that student competence is increasing. It is important to acknowledge that the student-teacher ratio is not the only factor at play in improving education and, thus, student competence. Multiple factors are at play; some are already known through research, and some still need to be discovered. We have shown that the student-teacher ratio is a crucial one.
