%!TEX root = ../report_template.tex
\section{Conclusion}


% This below may already be conclusion

% Todo better citation
For every other of the 16 federal states, there is a very strong positive correlation, for the number of repeaters. This can be interpreted as meaning that the sufficient availability of teachers not only increases grades but is especially beneficial for challenged students. In contrast, a higher budget only helps when the schools can find teachers to employ. Making sure that many teachers are available is one of the most important challenges for the education system. The prognosis of the Kultusministerkonferenz \cite{kultusminister_konferenz_lehrkrafteeinstellungsbedarf_2023} shows that there are still more open positions than teachers that can fill them. Unfortunately, they predict that this gap will eventually close in the coming decade. This means that a further increase in grades in the future can be expected.

It is important to note that having enough teachers is not the only factor responsible for the rising Abitur grades. Nonetheless, it is one of the most important ones. While the German education system faces several challenges, our demonstration illustrates that it has effectively addressed certain issues over the past decade and is poised to continue resolving them in the future. The increasing grades are a result of an increase in the competence of the students, facilitated by an improvement in the education system, especially a decrease in the student-teacher ratio.

% We have introduced a new approach to explaining the increasing Abitur grades. There is a very strong correlation between the student-teacher ratio and the Abitur grades. Additionally, we also found a negative correlation between this ratio and the repeater number. This means that the number of teachers not only has a positive impact on the grades but also on the more challenged students. Improving the budget does not necessarily help. It is essential for the German education system to have enough teachers available.

% Todo Reuse 
% Grade inflation in the Abitur grades has not been scientifically proven so far. What has been proven is that student competence is increasing. It is important to acknowledge that the student-teacher ratio is not the only factor at play in improving education and, thus, student competence. Multiple factors are at play; some are already known through research, and some still need to be discovered. We have shown that the student-teacher ratio is a crucial one.
