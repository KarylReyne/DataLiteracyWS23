%!TEX root = ../report_template.tex
\section*{Methods}
%Todo: Think about citations
The basis for the analysis of the German school system are the average final Abitur grades. The grades are published every year by the \citeauthor{kultusminister_konferenz_abiturnoten_nodate}. Each file contains the count of children per written grade and federal state. The grades are defined in $0.1$ steps, with $4.0$ as the worst and $1.0$ as the best grade. Furthermore, the amount of children who failed with a grade worse than $4.0$ is aggregated in an additional column.

The second dataset is provided in the \textit{Fachreport Schuljahr 2020/21} of the \citeauthor{statistische_bundesamt_allgemeinbildende_2022} and contains the number of teachers from 1992 until 2020. The dataset groups them primarily according to their contract type, federal state, and school type. This paper merges the teacher counts with two student datasets, which are published in the \textit{Genesis} database of the \citeauthor{statistische_bundesamt_statistisches_2023}. Both provide the number of students as different groupings and aggregations. The first contains the number of children per grade and school type  for the years 1998 to 2022. In contrast, the second table provides the absolute amount of children, leavers, and beginners in each federal state from 1997 to 2022. Therefore, the analysis of the merged dataset can only be conducted separately for school types and federal states.