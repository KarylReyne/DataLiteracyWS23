%!TEX root = ../report_template.tex
\section{Introduction} \label{sec:Introduction}
The Abitur grades have constantly increased in the German school system over the past decades. This aspect becomes evident in the course of this analysis. Every year, when the Abitur takes place, the grades and the difficulty of the exercises are extensively discussed in the media and have been part of a fierce research discussion for decades. The central focus of the discourse revolves around the question of whether grade inflation occurs, signifying a rise in grades without a corresponding increase in competence or knowledge.


The discourse has predominantly centred around mathematics, since the difficulty of exercises is easiest to compare. The line is drawn between mathematicians arguing that specific exercises are easier than exercises in the past \cite{kuhnel_modellierungskompetenz_2015, jahnke_hamburger_2014,lemmermeyer_zentralabitur_nodate} and studies claiming that grade inflation cannot be reliably proven since the competence of students has also increased \cite{schleithoff_noteninflation_2015}.
In 2015, a data-driven approach was employed, involving the analysis of comprehensive data on the education system. The analysis was promising, but not yet enough to neglect the claim of a grade inflation \cite{grozinger_gibt_2015}.

This paper expands on that work, disproving the claim that grade inflation is the main cause of the observed trend. An explanatory framework for the improvement of Abitur grades is provided by this data analysis, building upon past research. All analyzed data is taken from official federal resources such as the German Federal Statistics Office.\footnote{A python framework for performing the data analysis discussed in this paper is available at \url{https://github.com/KarylReyne/DataLiteracyWS23}}

The study undertakes an analysis of the quantifiable impacts stemming from the upward trend in Abitur grades within the educational system. A new influential factor, namely the student-teacher ratio will be discovered and analysed. Prognostications are offered concerning the trajectory of future developments and the implications for the German education system.
