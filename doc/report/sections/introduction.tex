\section{Introduction}
(Motivation)
The Abitur grades have constantly increased in the german school system over the past decades. Every year, when the Abitur takes place, the grades and the appropriateness of difficulty of exercises is the hot topic in media. \\
There is a high desire to criticize the german school system, political decisions and the comission responsible for conducting the exam. It is a fertile ground for loosely justified speculations. One thesis that comes up every year is that the Abitur is getting easier. (Zeitungartikel)\\\\
(Current research)
Is the thesis "the Abitur is getting easier" justified and can it be supported with data? It is really hard to supported that claim by looking at the exercises, since difficulty is not measurable and somewhat subjective. There are multiple studies looking at specific exercises and comparing them to past exercises. Epsecially the difficulty of the math exercises are heavily criticized \cite{kuhnel2015modellierungskompetenz} \cite{JahnkeKleinKühnelSonarSpindler+2014+115+122} \cite{lemmermeyer2019zentralabitur}. For the reasons above, this approach is highly controversial. Other reasons for grade inflation are discussed in multiple papers. Many are speculative. Still, the improving grades are probably caused by multiple factors.\\\\
In this paper we argue, that the justifications for the claim that the Abitur is getting easier over the years, are barely supported by actual data. The various arguments made are usually based on subjective opinions and experiences.
(Topic of the paper)
This paper examines claims regarding the factors contributing to the improvement of Abitur grades and inverstigates which of these claims are supported by the data. This data analysis provide an explanatory framework for the improvement of Abitur grades by analysing data mainly from the German Federal Statitical Office and other federal ressources.\\
Further we will investigate what the measurable effects of the increasing Abitur grades are and if those effects are positive or negative. Finally we will make a prediction for the grade development of the future and the future of the german school system.
