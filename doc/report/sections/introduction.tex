\section{Introduction}
The Abitur grades have constantly increased in the german school system over the past decades. Every year, when the Abitur takes place, the grades and the appropriateness of difficulty of exercises is extensively discussed in the media and has been part of a fierce research discussion for decades. The central focus of the discourse revolves around the question of whether grade inflation occurs, signifying a rise in grades without a corresponding increase in competence or knowledge. \\\\
The discourse has predominantly centered around mathematics, since the difficulty of exercises is easiest to compare. The line is drawn between mathematicians arguing that specific exercises are easier than exercises in the past \cite{kuhnel2015modellierungskompetenz} \cite{JahnkeKleinKühnelSonarSpindler+2014+115+122} \cite{lemmermeyer2019zentralabitur} and studies claiming that grade inflation cannot be reliably proven, since competence of students has also increased \cite{Schleithoff+2015+3+26}. 
In 2015, a data-driven approach was employed, involving the analysis of comprehensive data on the education system. The analysis was promising, but not yet enough to neglect the claim of a grade inflation. \cite{doi:10.7788/bue-2015-0407}.\\\\
This paper expands on that work, disproving the claim that grade inflation is happening. The conducted data analysis, builds upon past research, to provide an explanatory framework for the improvement of Abitur grades. The anylysed data is taken from official ressources such as the German Federal Statitical Office. \\
The study undertakes an analysis of the quantifiable impacts stemming from the upward trend in Abitur grades within the educational system.
Prognostications are offered concerning the trajectory of future grade developments and the implications for the German school system.
