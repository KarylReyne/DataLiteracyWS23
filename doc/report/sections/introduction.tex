\section{Introduction}
(Motivation)
The Abitur grades have constantly increased in the german school system over the past decades. Every year, when the Abitur takes place, the grades and the appropriateness of difficulty of exercises is the hot topic in media. \\
There is a high desire to critizize the german school system, political decisions and the comission responsible for conducting the exam. It is a fertile ground for loosely justified speculations. One thesis that comes up every year is that the Abitur is getting easier. (Zeitungartikel)\\\\
(Current research)
Is the thesis "the Abitur is getting easier" justified and can it be supported with data? The reasons for grade inflation are discussed in multiple papers (cite papers and describe their content) and there is no clear cut resolution. The causes are various.\\\\
(Topic of the paper)
In this paper we argue, that the justifications for the claim that the Abitur is getting easier over the years, are barely supported by actual data. The various arguments made are usually based on subjective opinions and experiences. (cite paper about people looking at the math exercises). The data is not supporting that claim. In contrast, there are causes that are supported by data. These causes will be investigated. Even though they might not tell the whole story, they provide a good explanatory framework, that is not based on opinions and speculation.\\
The measurable correlations will be explained and supported by data analysis of data from the German Federal Statitical Office and other federal sources. We will proove that especially the increased financing of the education system has a positive correlation with theAbitur grades.\\
Further we will investigate what the measurable effects of the increasing Abitur grades are and if those effects are positive or negative. Finally we will make a prediction for the grade development of the future and the future of the german school system.
