%!TEX root = ../report_template.tex
\section{Introduction} \label{sec:Introduction}
The Abitur grades have constantly increased in the German school system over the past years \cite{kultusminister_konferenz_abiturnoten_nodate}, resulting in a research discussion in the media (todo source). The main topic of discussion is whether grades get better, even if students performance is worse.

The discourse has predominantly centered around mathematics, since the difficulty of exercises is easiest to compare. On the one hand-side, mathematicians argue that Germany has a grade inflation, resulting in easier exercises over the years \cite{kuhnel_modellierungskompetenz_2015, jahnke_hamburger_2014,lemmermeyer_zentralabitur_nodate}. On the other hand-side, there are studies claiming that grade inflation cannot be reliably proven since the competence of students has also increased \cite{schleithoff_noteninflation_2015}. In \citeyear{grozinger_gibt_2015}, a data-driven approach was employed by \citeauthor{grozinger_gibt_2015}, involving the analysis of comprehensive data on the education system. The results were promising, but not yet enough to dismiss the claim of grade inflation.

This paper expands on that work, attempting to disprove the claim that grade inflation is the only cause of the observed trend. Building upon past research, a data analysis is conducted that aims to provide insight into the causes and effects on the schoolchildren's performance. \footnote{Source files are publicly available in the \href{https://github.com/KarylReyne/DataLiteracyWS23}{project repository}.}.

Furthermore, this study investigates whether there are any quantifiable causes of this upward trend in Abitur grades in the German education system. In addition, the causes and evolution of the repeater ratio across all school types are investigated to project the results to other educational institutions. Finally, prognoses about the trajectory of future grade developments and their implications for the German education system are outlined.