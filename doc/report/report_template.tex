\documentclass{article}

% Recommended, but optional, packages for figures and better typesetting:
\usepackage{microtype}
\usepackage{graphicx}
\usepackage{subfigure}
\usepackage{booktabs} % for professional tables

\usepackage{tikz}
% Corporate Design of the University of Tübingen
% Primary Colors
\definecolor{TUred}{RGB}{165,30,55}
\definecolor{TUgold}{RGB}{180,160,105}
\definecolor{TUdark}{RGB}{50,65,75}
\definecolor{TUgray}{RGB}{175,179,183}

% Secondary Colors
\definecolor{TUdarkblue}{RGB}{65,90,140}
\definecolor{TUblue}{RGB}{0,105,170}
\definecolor{TUlightblue}{RGB}{80,170,200}
\definecolor{TUlightgreen}{RGB}{130,185,160}
\definecolor{TUgreen}{RGB}{125,165,75}
\definecolor{TUdarkgreen}{RGB}{50,110,30}
\definecolor{TUocre}{RGB}{200,80,60}
\definecolor{TUviolet}{RGB}{175,110,150}
\definecolor{TUmauve}{RGB}{180,160,150}
\definecolor{TUbeige}{RGB}{215,180,105}
\definecolor{TUorange}{RGB}{210,150,0}
\definecolor{TUbrown}{RGB}{145,105,70}

% hyperref makes hyperlinks in the resulting PDF.
% If your build breaks (sometimes temporarily if a hyperlink spans a page)
% please comment out the following usepackage line and replace
% \usepackage{icml2023} with \usepackage[nohyperref]{icml2023} above.
\usepackage{hyperref}


% Attempt to make hyperref and algorithmic work together better:
\newcommand{\theHalgorithm}{\arabic{algorithm}}

\usepackage[accepted]{icml2023}

% For theorems and such
\usepackage{amsmath}
\usepackage{amssymb}
\usepackage{mathtools}
\usepackage{amsthm}

% if you use cleveref..
\usepackage[capitalize,noabbrev]{cleveref}

%%%%%%%%%%%%%%%%%%%%%%%%%%%%%%%%
% THEOREMS
%%%%%%%%%%%%%%%%%%%%%%%%%%%%%%%%
\theoremstyle{plain}
\newtheorem{theorem}{Theorem}[section]
\newtheorem{proposition}[theorem]{Proposition}
\newtheorem{lemma}[theorem]{Lemma}
\newtheorem{corollary}[theorem]{Corollary}
\theoremstyle{definition}
\newtheorem{definition}[theorem]{Definition}
\newtheorem{assumption}[theorem]{Assumption}
\theoremstyle{remark}
\newtheorem{remark}[theorem]{Remark}

% Todonotes is useful during development; simply uncomment the next line
%    and comment out the line below the next line to turn off comments
%\usepackage[disable,textsize=tiny]{todonotes}
\usepackage[textsize=tiny]{todonotes}

\newcommand{\D}{\text{d}}% leibniz notation d for derivatives/integrals
\renewcommand{\qed}{\hfill \ensuremath{\Box}}% white qed box aligned to the right
\newcommand{\lp}[1]{\left({#1}\right)}% large parentheses, e.g. for vectors or fractions
\newcommand{\mono}{\texttt}% monospace text
\newcommand{\ceil}[1]{\left\lceil{#1}\right\rceil}
\newcommand{\floor}[1]{\left\lfloor{#1}\right\rfloor}
\newcommand{\cont}{\lightning}

\icmltitlerunning{Project Report for Data Literacy 2023/24}

\begin{document}

\twocolumn[
\icmltitle{
    Project Report for Data Literacy 2023/24\\ 
    Impact of the Covid-19 Pandemic on the German School System
}

\icmlsetsymbol{equal}{*}

\begin{icmlauthorlist}
\icmlauthor{Jonathan Schwab}{equal,first}
\icmlauthor{Lars Kasüschke}{equal,second}
\icmlauthor{Niklas Munkes}{equal,third}
\icmlauthor{Tom Freudenmann}{equal,fourth}
\end{icmlauthorlist}

\icmlaffiliation{first}{Matrikelnummer 6765897, jonathan.schwab@student.uni-tuebingen.de, MSc Computer Science}
\icmlaffiliation{second}{Matrikelnummer 4247775, lars.kasueschke@student.uni-tuebingen.de, BSc Computer Science}
\icmlaffiliation{third}{Matrikelnummer 4269436, niklas.munkes@student.uni-tuebingen.de, MSc Media Informatics}
\icmlaffiliation{fourth}{Matrikelnummer 6631101, tom.freudenmann@student.uni-tuebingen.de, MSc Computer Science}

\vskip 0.3in
]
\printAffiliationsAndNotice{\icmlEqualContribution}

\begin{abstract}
TODO
Test
\end{abstract}

\section{Introduction}
(Motivation)
The Abitur grades have constantly increased in the german school system over the past decades. Every year, when the Abitur takes place, the grades and the appropriateness of difficulty of exercises is the hot topic in media. \\
There is a high desire to criticize the german school system, political decisions and the comission responsible for conducting the exam. It is a fertile ground for loosely justified speculations. One thesis that comes up every year is that the Abitur is getting easier. (Zeitungartikel)\\\\
(Current research)
Is the thesis "the Abitur is getting easier" justified and can it be supported with data? It is really hard to supported that claim by looking at the exercises, since difficulty is not measurable and somewhat subjective. Math has been the main battleground of the discussion. There are multiple studies looking at specific exercises and comparing them to past exercises \cite{kuhnel2015modellierungskompetenz} \cite{JahnkeKleinKühnelSonarSpindler+2014+115+122} \cite{lemmermeyer2019zentralabitur}. For the reasons above, this approach is highly controversial. On the other side multiple papers also argue that the overall competence of students has been increased, so a grade deflation can not be proven \cite{Schleithoff+2015+3+26}\\ Other discussed reasons for grade inflation are Many are speculative. Still, the improving grades are probably caused by multiple factors.\\\\
In this paper we argue, that the justifications for the claim that the Abitur is getting easier over the years, are barely supported by actual data. The various arguments made are usually based on subjective opinions and experiences.
(Topic of the paper)
This paper examines claims regarding the factors contributing to the improvement of Abitur grades and inverstigates which of these claims are supported by the data. This data analysis provide an explanatory framework for the improvement of Abitur grades by analysing data mainly from the German Federal Statitical Office and other federal ressources.\\
Further we will investigate what the measurable effects of the increasing Abitur grades are and if those effects are positive or negative. Finally we will make a prediction for the grade development of the future and the future of the german school system.


\bibliography{bibliography}
\bibliographystyle{icml2023}

\end{document}
